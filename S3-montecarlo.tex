\section{Monte Carlo Results}\label{sec:2}
This section presents Monte Carlo experiments demonstrating that
this paper's modified version of \citepos{ClW:07} statistic performs
similarly to their original test in the situations they study, but can
have substantially higher power when the \dgp\ has a structural
break.%
\footnote{All of these simulations were programmed in R
  \citep[version 2.14.0]{R} and use the \allcaps{MASS}
  \citep[7.3-22]{VeR:02} package.} %

The \dgp\ has three different parametrizations: one to study the
tests' size, one to study power under stationarity, and one to study
power if there is a single break in the relationship between the
target and predictors.  The \dgp\ is:
\begin{align*}
  y_{t+1} &= \gamma_{1t} + \gamma_{2t} x_{t} + \ep_{t+1} &
  \gamma_t &=
  \begin{cases}
    (0.5, 0)    & \text{size simulations} \\
    (0.5, 0.35) & \text{power (stable)} \\
    (-0.5, 0)    & t \leq \tfrac{T}{2} \quad \text{power (break)} \\
    (1, 0.35) & t > \tfrac{T}{2} \quad \text{power (break)}
  \end{cases}\\\nonumber
  x_{t+1} &= 0.15 + 0.95 x_{t} + u_{t+1} &
  (\ep_t, u_t)' &\sim iid\ N\Bigg(\begin{pmatrix} 0 \\ 0
  \end{pmatrix}
   , \begin{pmatrix} 18 & -
    0.5 \\ -0.5 & 0.025 \end{pmatrix}\Bigg)
  \\ R &= 120, 240 & P &= 120, 240, 360, 720.
\end{align*}
Both models are estimated by \ols. The benchmark model regresses $y_{t+1}$
on a constant, and the alternative regresses $y_{t+1}$ on a constant and
$x_t$.  \citet{ClW:07} argue that this \dgp\ mimics an asset
pricing application similar to \citepos{GoW:08} which we study in
Section~\ref{sec:3}.

For comparison, we study this paper's new statistic as well as
\poscw\ rolling-window and recursive-window test statistics.  Clark and West
only prove that their rolling-window statistic is asymptotically
normal, and only then if the benchmark model is not estimated, but
their recursive-window statistic is popular in practice and in
simulations tends to perform similarly to their rolling window test.
We use all three of these statistics to test the null that the
benchmark model's innovation is an \mds.%
\footnote{\citet{ClW:07}
  report the performance of the tests proposed by \citet{CCS:01} and
  \citet{ClM:05} as well, and of tests based on the naive Gaussian
  statistic.} %

\begin{table}[tb]
  \centering
  % latex table generated in R 3.2.0 by xtable 1.8-0 package
% Tue Feb  2 03:47:02 2016
\begin{tabularx}{\textwidth}{lCCCCC}
  \toprule Sim. type & R & P & Pr[CW~roll.] & Pr[CW~rec.] & Pr[new] \\ 
  \midrule size & $120$ & $120$ & $\enskip7.3$ & $\enskip7.8$ & $\enskip\enskip7.6$ \\ 
   &  & $240$ & $\enskip5.5$ & $\enskip5.5$ & $\enskip\enskip6.2$ \\ 
   &  & $360$ & $\enskip7.5$ & $\enskip6.2$ & $\enskip\enskip7.7$ \\ 
   &  & $720$ & $\enskip8.5$ & $\enskip5.4$ & $\enskip\enskip7.2$ \\ 
   & $240$ & $120$ & $\enskip7.2$ & $\enskip7.2$ & $\enskip\enskip7.7$ \\ 
   &  & $240$ & $\enskip6.3$ & $\enskip6.5$ & $\enskip\enskip7.1$ \\ 
   &  & $360$ & $\enskip6.8$ & $\enskip5.9$ & $\enskip\enskip6.8$ \\ 
   &  & $720$ & $\enskip7.0$ & $\enskip5.9$ & $\enskip\enskip7.3$ \\ 
  power (stable) & $120$ & $120$ & $26.2$ & $30.0$ & $\enskip29.2$ \\ 
   &  & $240$ & $39.2$ & $47.2$ & $\enskip42.4$ \\ 
   &  & $360$ & $47.3$ & $59.8$ & $\enskip51.1$ \\ 
   &  & $720$ & $66.8$ & $82.3$ & $\enskip73.1$ \\ 
   & $240$ & $120$ & $34.5$ & $36.1$ & $\enskip34.1$ \\ 
   &  & $240$ & $45.9$ & $50.1$ & $\enskip46.9$ \\ 
   &  & $360$ & $56.7$ & $63.8$ & $\enskip56.9$ \\ 
   &  & $720$ & $78.2$ & $87.0$ & $\enskip78.7$ \\ 
  power (breaks) & $120$ & $120$ & $25.9$ & $29.9$ & $\enskip62.2$ \\ 
   &  & $240$ & $30.1$ & $31.0$ & $\enskip87.4$ \\ 
   &  & $360$ & $35.5$ & $32.9$ & $\enskip96.5$ \\ 
   &  & $720$ & $46.1$ & $38.2$ & $\enskip99.8$ \\ 
   & $240$ & $120$ & $28.1$ & $30.6$ & $\enskip58.2$ \\ 
   &  & $240$ & $37.6$ & $36.1$ & $\enskip87.7$ \\ 
   &  & $360$ & $43.1$ & $39.0$ & $\enskip97.2$ \\ 
   &  & $720$ & $56.9$ & $42.5$ & $100.0$ \\ 
   \bottomrule \end{tabularx}

  \caption{Size and power of the \oos\ tests in the simulations
    described by Section~\ref{sec:2}, at
    \testsize\% confidence.  These percentages are calculated from \totalsims\
    samples.  Pr[\allcaps{CW} roll.] shows the fraction of simulations for
    which Clark and West's (2007) rolling-window statistic rejects;
    Pr[\allcaps{CW} rec.] shows the fraction of simulations for which
    their recursive-window statistic rejects; and Pr[new] shows the fraction of
    simulations for which this paper's test rejects.}
\label{tab:mc1}
\end{table}

Table~\ref{tab:mc1} presents the simulation results.  For all of the
stable parameter values, the proposed new statistic has similar
rejection probability to \citepos{ClW:07}.  Both of Clark and West's
tests are generally slightly undersized relative to our new test,
which is itself slightly undersized: when $R$ is 120 and $P$ is 360
our test statistic has size 7.6\% and Clark and West's rolling and
recursive window tests have size 7.5\% and 6.2\% respectively, at a
nominal size of 10\%.  For the stable alternative, our new statistic
typically has slightly higher power than Clark and West's rolling
window and lower power than their recursive window.  For example, when
$R$ is 120 and $P$ is 720, the rolling-window test rejects at 66.8\%,
our statistic at 73.0\%, and the recursive window statistic at 82.3\%,
again for a nominal size of 10\%.  In general, the statistics perform
similarly under stability.

For the simulations with a single break, the new statistic has
considerably higher power than \poscw\ original tests across all of
the choices of $R$ and $P$; the rejection probability is more than
twice as large for most parametrizations.  When $R$ is 120 and $P$ is
360 with a nominal size of 10\%, for example, the new statistic
rejects at 96.4\% while the rolling and recursive window statistics
reject at 35.5\% and 32.9\% respectively.  Results for other choices
of nominal size and sample split give similar results.  So mixing
window strategies can give a large power advantage when testing for
time-varying predictability, and performs similarly to the original
test when testing for stable outperformance.

%%% Local Variables:
%%% mode: latex
%%% TeX-master: "mixedwindow"
%%% TeX-command-extra-options: "-shell-escape"
%%% End:
